\section{手順}
\subsection{課題1}
recycleRobotMDP.csvの表\ref{table:1}を参考にして遷移確率や報酬を変更したcase2, case3を作
成する。作成した報酬の与え方がエージェントの目標になっていることを示す。
\begin{table}[hbtp]
  \begin{minipage}[t]{\hsize}
  \centering
  \caption{元のMDP(case1)}
  \label{table:1}
    \begin{tabular}{|c|c|c|c|c|}
      \hline
      状態 & 行動 & 次状態 & 遷移確率 & 報酬\\
      \hline
      \hline
      START & N & high & 1 & 0 \\
      \hline
      high & search & high & 0.8 & 12 \\
      high & search & low & 0.2 & 4 \\
      \hline
      low & search & GOAL & 0.1 & -3 \\
      low & search & low & 0.9 & 2 \\
      \hline
      high & wait & high & 1 & 1 \\
      high & wait & low & 0 & 1 \\
      \hline
      low & wait & GOAL & 0 & 1 \\
      low & wait & low & 1 & 1 \\
      \hline
      low & recharge & GOAL & 1 & 0 \\
      low & recharge & low & 0 & 0 \\
      \hline
      GOAL & N & high & 1 & 0 \\
      \hline
    \end{tabular}
  \end{minipage}
\end{table}

\begin{enumerate}
  \item case2,case3をそれぞれ以下のように作成した。
  \begin{itemize}
    \item[case2] 表\ref{table:2}のように、遷移確率は変更せず報酬だけを変更した。
    報酬の決め方は、以下の点を意識した。
    \begin{itemize}
      \item 行動「recharge」以外で状態「GOAL」になるということは、他の存在に助けてもらう必要がある。
      そのため、リサイクルロボット単体で完結するように、行動「recharge」以外で次状態が「GOAL」になる全ての行動の報酬を「-7」にした。
      \item 上記の理由で、行動「recharge」の報酬を「7」にした。
      \item リサイクルロボットに積極的に「search」してもらいたいため、行動「wait」の報酬を「-1」にした。
    \end{itemize}
    \item[case3] 表\ref{table:3}のように、遷移確率を変更して、時間経過でリサイクルロボットのバッテリーが減るようにした。
    報酬の決め方は、以下の点を意識した。
    \begin{itemize}
      \item リサイクルロボットがバッテリーを無駄なく使えるように、バッテリーが減る場合の報酬を行動ごとにそれぞれ減らした。
      \item 行動「recharge」以外で状態「GOAL」になるということは、他の存在に助けてもらう必要がある。
      そのため、リサイクルロボット単体で完結するように、行動「recharge」以外で次状態が「GOAL」になる全ての行動の報酬を「-7」にした。
      \item 上記の理由で、行動「recharge」の報酬を「7」にした。
      \item リサイクルロボットに積極的に「search」してもらいたいため、行動「wait」の報酬を「-1」にした。
    \end{itemize}
  \end{itemize}
  
  \begin{table}
    \begin{minipage}[t]{0.45\hsize}
      \centering
      \caption{作成したMDP(case2)}
      \label{table:2}
        \begin{tabular}{|c|c|c|c|c|}
          \hline
          状態 & 行動 & 次状態 & 遷移確率 & 報酬\\
          \hline
          \hline
          START & N & high & 1 & 0 \\
          \hline
          high & search & high & 0.8 & 10 \\
          high & search & low & 0.2 & 4 \\
          \hline
          low & search & GOAL & 0.1 & -7 \\
          low & search & low & 0.9 & 2 \\
          \hline
          high & wait & high & 1 & -1 \\
          high & wait & low & 0 & 0 \\
          \hline
          low & wait & GOAL & 0 & 0 \\
          low & wait & low & 1 & -1 \\
          \hline
          low & recharge & GOAL & 1 & 7 \\
          low & recharge & low & 0 & 0 \\
          \hline
          GOAL & N & high & 1 & 0 \\
          \hline
        \end{tabular}
      \end{minipage}
      \begin{minipage}[t]{0.45\hsize}
        \centering
        \caption{作成したMDP(case3)}
        \label{table:3}
          \begin{tabular}{|c|c|c|c|c|}
            \hline
            状態 & 行動 & 次状態 & 遷移確率 & 報酬\\
            \hline
            \hline
            START & N & high & 1 & 0 \\
            \hline
            high & search & high & 0.8 & 10 \\
            high & search & low & 0.2 & 4 \\
            \hline
            low & search & GOAL & 0.1 & -7 \\
            low & search & low & 0.9 & 2 \\
            \hline
            high & wait & high & 0.9 & -1 \\
            high & wait & low & 0.1 & -4 \\
            \hline
            low & wait & GOAL & 0.1 & -7 \\
            low & wait & low & 0.9 & -1 \\
            \hline
            low & recharge & GOAL & 0.8 & 7 \\
            low & recharge & low & 0.2 & -1 \\
            \hline
            GOAL & N & high & 1 & 0 \\
            \hline
          \end{tabular}
        \end{minipage}
  \end{table}
  
  \item 以下に、case2とcase3の結果を示す。
  \begin{table}
    \begin{minipage}[t]{0.45\hsize}
      \centering
      \caption{case2の結果}
      \label{table:4}
      \begin{tabular}{|c|c|c|c|}
        \hline
        状態 & 行動 & Q値 & Q値更新回数\\
        \hline
        GOAL & N & 0 & 0 \\
        START & N & 0 & 0 \\
        high & search & 11.194368 & 54 \\
        high & wait & 4.146994 & 54 \\
        low & recharge & 4.310171 & 94 \\
        low & search & 1.989767 & 94 \\
        low & wait & 1.20334 & 94 \\
        \hline
      \end{tabular}
    \end{minipage}
    \begin{minipage}[t]{0.45\hsize}
      \centering
      \caption{case3の結果}
      \label{table:5}
        \begin{tabular}{|c|c|c|c|}
          \hline
          状態 & 行動 & Q値 & Q値更新回数\\
          \hline
          GOAL & N & 0 & 0 \\
          START & N & 0 & 0 \\
          high & search & 9.995437 & 58 \\
          high & wait & 3.695292 & 58 \\
          low & recharge & 3.895574 & 109 \\
          low & search & 1.728605 & 109 \\
          low & wait & 0.750725 & 109 \\
          \hline
        \end{tabular}
      \end{minipage}
  \end{table}



\end{enumerate}



\subsection{課題2}
\subsubsection{計測対象}
