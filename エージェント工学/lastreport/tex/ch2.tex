\section{結果}
\subsection{課題1}
\begin{itemize}
  \item 課題1の出力結果は表\ref{table:1}~表\ref{table:3}のようになった。
  \begin{itemize}
    \item[→] 表\ref{table:1}を見て分かる通り、処理を1回だけ計測した場合は、最初の2回分の処理時間が長くなっている。
    それとは異なり、表\ref{table:2}と表\ref{table:3}は、最初から処理時間が安定しているように見える。

  \end{itemize}
  \begin{table}[hbtp]
    \begin{tabular}{ccc}
      \begin{minipage}[t]{0.30\hsize}
      \centering
      \caption{課題1の計測結果}
      \label{table:1}
        \begin{tabular}{|l|rl|}
          \hline
          試行回数 & \multicolumn{2}{c|}{処理時間}\\
          \hline
          1回目 & 28 & ms \\
          2回目 & 25 & ms \\
          3回目 & 17 & ms \\
          4回目 & 15 & ms \\
          5回目 & 15 & ms \\
          6回目 & 15 & ms \\
          7回目 & 15 & ms \\
          8回目 & 15 & ms \\
          9回目 & 15 & ms \\
          10回目 & 15 & ms \\
          11回目 & 15 & ms \\
          12回目 & 15 & ms \\
          13回目 & 15 & ms \\
          14回目 & 16 & ms \\
          15回目 & 15 & ms \\
          16回目 & 15 & ms \\
          17回目 & 15 & ms \\
          18回目 & 15 & ms \\
          19回目 & 15 & ms \\
          20回目 & 14 & ms \\
          \hline
        \end{tabular}
      \end{minipage}
      \begin{minipage}[t]{0.33\hsize}
        \centering
        \caption{10回分を計測して10で割った結果}
        \label{table:2}
          \begin{tabular}{|l|rl|}
            \hline
            試行回数 & \multicolumn{2}{c|}{処理時間}\\
            \hline
            1回目 & 17 & ms \\
            2回目 & 15 & ms \\
            3回目 & 15 & ms \\
            4回目 & 15 & ms \\
            5回目 & 15 & ms \\
            6回目 & 15 & ms \\
            7回目 & 15 & ms \\
            8回目 & 16 & ms \\
            9回目 & 16 & ms \\
            10回目 & 16 & ms \\
            11回目 & 16 & ms \\
            12回目 & 16 & ms \\
            13回目 & 15 & ms \\
            14回目 & 15 & ms \\
            15回目 & 15 & ms \\
            16回目 & 15 & ms \\
            17回目 & 15 & ms \\
            18回目 & 15 & ms \\
            19回目 & 15 & ms \\
            20回目 & 15 & ms \\
            \hline
          \end{tabular}
        \end{minipage}
        \begin{minipage}[t]{0.33\hsize}
          \centering
          \caption{20回分を計測して20で割った結果}
          \label{table:3}
            \begin{tabular}{|l|rl|}
              \hline
              試行回数 & \multicolumn{2}{c|}{処理時間}\\
              \hline
              1回目 & 16 & ms \\
              2回目 & 15 & ms \\
              3回目 & 15 & ms \\
              4回目 & 15 & ms \\
              5回目 & 15 & ms \\
              6回目 & 15 & ms \\
              7回目 & 15 & ms \\
              8回目 & 15 & ms \\
              9回目 & 15 & ms \\
              10回目 & 15 & ms \\
              11回目 & 15 & ms \\
              12回目 & 15 & ms \\
              13回目 & 15 & ms \\
              14回目 & 16 & ms \\
              15回目 & 15 & ms \\
              16回目 & 16 & ms \\
              17回目 & 15 & ms \\
              18回目 & 15 & ms \\
              19回目 & 15 & ms \\
              20回目 & 15 & ms \\
              \hline
            \end{tabular}
          \end{minipage}
    \end{tabular}
  \end{table}
  \clearpage
\item 課題1のプログラムを分割して計測した結果は表\ref{table:6}のようになった。
\begin{itemize}
  \item[→] 表を見てわかる通り、変数の作成には、ほとんど時間は使われておらず、ほぼすべての時間が画像処理に使われている。
\end{itemize}
  \begin{table}[hbtp]
    \centering
      \caption{課題1のプログラムを分割して計測した結果}
      \label{table:6}
        \begin{tabular}{|l|c|c|}
          \hline
          試行回数 & 変数作成時間 &処理時間\\
          \hline
          1回目 & 0ms & 31ms \\
          2回目 & 0ms & 39ms \\
          3回目 & 1ms & 16ms \\
          4回目 & 0ms & 17ms \\
          5回目 & 0ms & 16ms \\
          6回目 & 0ms & 16ms \\
          7回目 & 0ms & 17ms \\
          8回目 & 0ms & 15ms \\
          9回目 & 0ms & 16ms \\
          10回目 & 0ms & 14ms \\
          11回目 & 1ms & 16ms \\
          12回目 & 0ms & 15ms \\
          13回目 & 0ms & 15ms \\
          14回目 & 0ms & 16ms \\
          15回目 & 1ms & 15ms \\
          16回目 & 0ms & 15ms \\
          17回目 & 1ms & 15ms \\
          18回目 & 0ms & 15ms \\
          19回目 & 1ms & 15ms \\
          20回目 & 0ms & 15ms \\
          \hline
        \end{tabular}
  \end{table}

\end{itemize}
\clearpage

\subsection{課題2}
\begin{itemize}
  \item 課題2の出力結果は表\ref{table:4}~表\ref{table:5}のようになった。
  \begin{itemize}
    \item[→] 表を見てわかる通り、ソートアルゴリズムをクイックソートに変えてからは、
    バブルソートを使っていた時に見られた最初の2回の処理時間が長くなる現象がなくなっている。
  \end{itemize}
  \item \begin{table}[hbtp]
    \begin{tabular}{ccc}
      \begin{minipage}[t]{0.45\hsize}
      \centering
      \caption{バブルソートを使った計測結果}
      \label{table:4}
        \begin{tabular}{|l|rl|}
          \hline
          試行回数 & \multicolumn{2}{c|}{処理時間}\\
          \hline
          1回目 & 28 & ms \\
          2回目 & 25 & ms \\
          3回目 & 17 & ms \\
          4回目 & 15 & ms \\
          5回目 & 15 & ms \\
          6回目 & 15 & ms \\
          7回目 & 15 & ms \\
          8回目 & 15 & ms \\
          9回目 & 15 & ms \\
          10回目 & 15 & ms \\
          11回目 & 15 & ms \\
          12回目 & 15 & ms \\
          13回目 & 15 & ms \\
          14回目 & 16 & ms \\
          15回目 & 15 & ms \\
          16回目 & 15 & ms \\
          17回目 & 15 & ms \\
          18回目 & 15 & ms \\
          19回目 & 15 & ms \\
          20回目 & 14 & ms \\
          \hline
        \end{tabular}
      \end{minipage}
      \begin{minipage}[t]{0.45\hsize}
        \centering
        \caption{クイックソートを使った計測結果}
        \label{table:5}
          \begin{tabular}{|l|rl|}
            \hline
            試行回数 & \multicolumn{2}{c|}{処理時間}\\
            \hline
            1回目 & 17 & ms \\
            2回目 & 15 & ms \\
            3回目 & 15 & ms \\
            4回目 & 15 & ms \\
            5回目 & 15 & ms \\
            6回目 & 15 & ms \\
            7回目 & 15 & ms \\
            8回目 & 16 & ms \\
            9回目 & 16 & ms \\
            10回目 & 16 & ms \\
            11回目 & 16 & ms \\
            12回目 & 16 & ms \\
            13回目 & 15 & ms \\
            14回目 & 15 & ms \\
            15回目 & 15 & ms \\
            16回目 & 15 & ms \\
            17回目 & 15 & ms \\
            18回目 & 15 & ms \\
            19回目 & 15 & ms \\
            20回目 & 15 & ms \\
            \hline
          \end{tabular}
        \end{minipage}
    \end{tabular}
  \end{table}
\end{itemize}
