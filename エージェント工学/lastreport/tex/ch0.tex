\section{目的}
リサイクルロボットについての強化学習サンプルプログラムを用いて、パラメータを変更すること
などで強化学習について考察する。
\subsection{課題1}
プログラム課題から一つを選び、1枚の画像を処理する時間を以下の点に気を付けながら計測する。
\begin{itemize}
  \setlength{\leftskip}{2.0cm}
  \item[注意点1] ‥計測対象は何かを示す。
  \item[注意点2] ‥計測方法は正確かどうかを調べる。
  \item[注意点3] ‥ループ回数を増減すると一回の処理当たりの所要時間が変化する理由を分析する。
\end{itemize}

\subsection{課題2}
任意のプログラム課題において、同じ処理でもアルゴリズムを変更すると、intやdoubleの計算を行う
回数が変化する。また、配列などのメモリに対するアクセス回数も変わり、オブジェクトの生成やメ
ソッドの呼び出しに時間がかかることなどにも注意して処理を比較する。
課題2においても、以下の点に注意しながら計測する。
\begin{itemize}
  \setlength{\leftskip}{2.0cm}
  \item[注意点1] ‥計測対象は何かを示す。
  \item[注意点2] ‥計測方法は正確かどうかを調べる。
  \item[注意点3] ‥アルゴリズムと処理時間の関係について分析する。
\end{itemize}