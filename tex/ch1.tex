\section{手順}
最初に、student\_id の部分に「801」を入力してデータを生成しておく。
\subsection{課題1}
\hyperlink{kbouhou2}{\hypertarget{kbouhou}{\subsubsection{k最近傍法}}}
パラメータを表\ref{table:1}のように設定して実行する。
\begin{table}[hbtp]
  \centering
  \caption{1.1.1のパラメータ}
  \label{table:1}
    \begin{tabular}{|l|c|}
      \hline
      パラメータ  & 値  \\
      \hline 
      train\_start  & 1 \\
      train\_end  & 400  \\
      train2\_start  & 0  \\
      train2\_end  &  0 \\
      test\_start  &  501 \\
      test\_end  &  10000 \\
      k\_NN  &  3 \\
      \hline
    \end{tabular}
\end{table}

\hyperlink{holda2}{\hypertarget{holda}{\subsubsection{ホールドアウト法(最初の80\%を学習・残りの20\%をテスト)}}}
\begin{enumerate}
  \item 全データ500の内、最初の80\%を学習データ、残りの20\%をテストデータにするため、400個を学習データに、100個をテストデータに分ける
  \item 上記のことから、1番目から400番目までのデータを学習データに、401番目から500番目のデータをテストデータにする。
  \item よって、パラメータを表\ref{table:2}のように設定して実行する。
\end{enumerate}

\begin{table}[hbtp]
  \centering
  \caption{1.1.2のパラメータ}
  \label{table:2}
    \begin{tabular}{|l|c|}
      \hline
      パラメータ  & 値  \\
      \hline
      train\_start  & 1 \\
      train\_end  & 400  \\
      train2\_start  & 0  \\
      train2\_end  &  0 \\
      test\_start  &  401 \\
      test\_end  &  500 \\
      k\_NN  &  3 \\
      \hline
    \end{tabular}
\end{table}
\clearpage

\hyperlink{holdb2}{\hypertarget{holdb}{\subsubsection{ホールドアウト法(最初の20\%をテスト・残りの80\%を学習)}}}
\begin{enumerate}
  \item 全データ500の内、最初の20\%をテストデータ、残りの80\%を学習データにするため、400個を学習データに、100個をテストデータに分ける
  \item 上記のことから、1番目から100番目までのデータをテストデータに、101番目から500番目のデータを学習データにする。
  \item よって、パラメータを表\ref{table:3}のように設定して実行する。
\end{enumerate}
\begin{table}[hbtp]
  \centering
  \caption{1.1.3のパラメータ}
  \label{table:3}
    \begin{tabular}{|l|c|}
      \hline
      パラメータ  & 値  \\
      \hline
      train\_start  & 101 \\
      train\_end  & 500  \\
      train2\_start  & 0  \\
      train2\_end  &  0 \\
      test\_start  &  1 \\
      test\_end  &  100 \\
      k\_NN  &  3 \\
      \hline
    \end{tabular}  
\end{table}

\hyperlink{gobun2}{\hypertarget{gobun}{\subsubsection{5分割の交差確認法}}}

\begin{enumerate}
  \item 全データ500を5分割して、それらを1つずつテストデータにする。
  \item その時の学習データには、それぞれの残りのデータを使う。
  \item よって、パラメータを表\ref{table:4}~表\ref{table:8}のように設定して、それぞれ実行する。
\end{enumerate}

\begin{table}[hbtp]
  \begin{tabular}{ccccc}
    \begin{minipage}{0.18\hsize}
      \begin{center}
        \caption{testdata:1~100}
        \label{table:4}
        \begin{tabular}{|l|c|}
          \hline
          パラメータ  & 値  \\
          \hline
          train\_start  & 101 \\
          train\_end  & 500  \\
          train2\_start  & 0  \\
          train2\_end  &  0 \\
          test\_start  &  1 \\
          test\_end  &  100 \\
          k\_NN  &  3 \\
          \hline
        \end{tabular}      
      \end{center}
    \end{minipage}
    
    \begin{minipage}{0.18\hsize}
      \begin{center}
        \caption{testdata:101~200}
        \label{table:5}
        \begin{tabular}{|l|c|}
          \hline
          パラメータ  & 値  \\
          \hline
          train\_start  & 1 \\
          train\_end  & 100  \\
          train2\_start  & 201  \\
          train2\_end  &  500 \\
          test\_start  &  101 \\
          test\_end  &  200 \\
          k\_NN  &  3 \\
          \hline
        \end{tabular}
      \end{center}
    \end{minipage}
    
    \begin{minipage}{0.18\hsize}
      \begin{center}
        \caption{testdata:201~300}
        \label{table:6}
        \begin{tabular}{|l|c|}
          \hline
          パラメータ  & 値  \\
          \hline
          train\_start  & 1 \\
          train\_end  & 200  \\
          train2\_start  & 301  \\
          train2\_end  &  500 \\
          test\_start  &  201 \\
          test\_end  &  300 \\
          k\_NN  &  3 \\
          \hline
        \end{tabular}
      \end{center}
    \end{minipage}
    
    \begin{minipage}{0.18\hsize}
      \begin{center}
        \caption{testdata:301~400}
        \label{table:7}
        \begin{tabular}{|l|c|}
          \hline
          パラメータ  & 値  \\
          \hline
          train\_start  & 1 \\
          train\_end  & 300  \\
          train2\_start  & 401  \\
          train2\_end  &  500 \\
          test\_start  &  301 \\
          test\_end  &  400 \\
          k\_NN  &  3 \\
          \hline
        \end{tabular}
      \end{center}
    \end{minipage}
    
    \begin{minipage}{0.18\hsize}
      \begin{center}
        \caption{testdata:401~500}
        \label{table:8}
        \begin{tabular}{|l|c|}
          \hline
          パラメータ  & 値  \\
          \hline
          train\_start  & 1 \\
          train\_end  & 400  \\
          train2\_start  & 0  \\
          train2\_end  &  0 \\
          test\_start  &  401 \\
          test\_end  &  500 \\
          k\_NN  &  3 \\
          \hline
        \end{tabular}
      \end{center}
    \end{minipage}
  \end{tabular}
\end{table}
\clearpage

\subsection{課題2}
\hyperlink{hyperpara2}{\hypertarget{hyperpara}{\subsubsection{最近傍法のハイパーパラメータの最適化}}}
\begin{enumerate}
  \item 全データ500の内、401番目から500番目をテストデータとする。
  \item 残りのデータ400から、学習データと検証データの比が3:1となるように分けると、それぞれ300個と100個のデータに分けられる。
  \item 上記より、学習データを1番目から300番目とし、検証データを301番目から400番目のデータとする。
  \item まずは、適当に選んだ5個以上の候補値にkの値を変えながら、検証データをテストに用いて、汎化能力を評価する。
  \item 上記より、パラメータを表\ref{table:9}~表\ref{table:17}のように設定して、それぞれ実行する。
  \item 実行した中で、評価値が一番高くなるkの値を見つける。
  \item ハイパーパラメータ最適化の結果の評価をするために、表\ref{table:18}のように、テストに用いるデータ
  を検証データからテストデータに変えた上で、学習に用いるデータに検証データも入れて、評価値が一番高いkの値を入れて再度実行する。
\end{enumerate}

\begin{table}[hbtp]
  \begin{tabular}{ccccc}
    \begin{minipage}{0.18\hsize}
      \begin{center}
        \caption{k=1}
        \label{table:9}
        \begin{tabular}{|l|c|}
          \hline
          パラメータ  & 値  \\
          \hline
          train\_start  & 1 \\
          train\_end  & 300  \\
          train2\_start  & 0  \\
          train2\_end  &  0 \\
          test\_start  &  301 \\
          test\_end  &  400 \\
          k\_NN  &  1 \\
          \hline
        \end{tabular}
      \end{center}
    \end{minipage}
    
    \begin{minipage}{0.18\hsize}
      \begin{center}
        \caption{k=6}
        \label{table:10}
        \begin{tabular}{|l|c|}
          \hline
          パラメータ  & 値  \\
          \hline
          train\_start  & 1 \\
          train\_end  & 300  \\
          train2\_start  & 0  \\
          train2\_end  &  0 \\
          test\_start  &  301 \\
          test\_end  &  400 \\
          k\_NN  &  6 \\
          \hline
        \end{tabular}
      \end{center}
    \end{minipage}
    
    \begin{minipage}{0.18\hsize}
      \begin{center}
        \caption{k=10}
        \label{table:11}
        \begin{tabular}{|l|c|}
          \hline
          パラメータ  & 値  \\
          \hline
          train\_start  & 1 \\
          train\_end  & 300  \\
          train2\_start  & 0  \\
          train2\_end  &  0 \\
          test\_start  &  301 \\
          test\_end  &  400 \\
          k\_NN  &  10 \\
          \hline
        \end{tabular}
      \end{center}
    \end{minipage}
    
    \begin{minipage}{0.18\hsize}
      \begin{center}
        \caption{k=17}
        \label{table:12}
        \begin{tabular}{|l|c|}
          \hline
          パラメータ  & 値  \\
          \hline
          train\_start  & 1 \\
          train\_end  & 300  \\
          train2\_start  & 0  \\
          train2\_end  &  0 \\
          test\_start  &  301 \\
          test\_end  &  400 \\
          k\_NN  &  17 \\
          \hline
        \end{tabular}
      \end{center}
    \end{minipage}
    
    \begin{minipage}{0.18\hsize}
      \begin{center}
        \caption{k=24}
        \label{table:13}
        \begin{tabular}{|l|c|}
          \hline
          パラメータ  & 値  \\
          \hline
          train\_start  & 1 \\
          train\_end  & 300  \\
          train2\_start  & 0  \\
          train2\_end  &  0 \\
          test\_start  &  301 \\
          test\_end  &  400 \\
          k\_NN  &  24 \\
          \hline
        \end{tabular}
      \end{center}
    \end{minipage}
  \end{tabular}
  
  \begin{tabular}{ccccc}
    \begin{minipage}{0.18\hsize}
      \begin{center}
        \caption{k=30}
        \label{table:14}
        \begin{tabular}{|l|c|}
          \hline
          パラメータ  & 値  \\
          \hline
          train\_start  & 1 \\
          train\_end  & 300  \\
          train2\_start  & 0  \\
          train2\_end  &  0 \\
          test\_start  &  301 \\
          test\_end  &  400 \\
          k\_NN  &  30 \\
          \hline
        \end{tabular}
      \end{center}
    \end{minipage}
    
    \begin{minipage}{0.18\hsize}
      \begin{center}
        \caption{k=38}
        \label{table:15}
        \begin{tabular}{|l|c|}
          \hline
          パラメータ  & 値  \\
          \hline
          train\_start  & 1 \\
          train\_end  & 300  \\
          train2\_start  & 0  \\
          train2\_end  &  0 \\
          test\_start  &  301 \\
          test\_end  &  400 \\
          k\_NN  &  38 \\
          \hline
        \end{tabular}
      \end{center}
    \end{minipage}
    
    \begin{minipage}{0.18\hsize}
      \begin{center}
        \caption{k=45}
        \label{table:16}
        \begin{tabular}{|l|c|}
          \hline
          パラメータ  & 値  \\
          \hline
          train\_start  & 1 \\
          train\_end  & 300  \\
          train2\_start  & 0  \\
          train2\_end  &  0 \\
          test\_start  &  301 \\
          test\_end  &  400 \\
          k\_NN  &  45 \\
          \hline
        \end{tabular}
      \end{center}
    \end{minipage}
    
    \begin{minipage}{0.18\hsize}
      \begin{center}
        \caption{k=52}
        \label{table:17}
        \begin{tabular}{|l|c|}
          \hline
          パラメータ  & 値  \\
          \hline
          train\_start  & 1 \\
          train\_end  & 300  \\
          train2\_start  & 0  \\
          train2\_end  &  0 \\
          test\_start  &  301 \\
          test\_end  &  400 \\
          k\_NN  &  52 \\
          \hline
        \end{tabular}
      \end{center}
    \end{minipage}
    
    \begin{minipage}{0.18\hsize}
      \begin{center}
        \caption{test run}
        \label{table:18}
        \begin{tabular}{|l|c|}
          \hline
          パラメータ  & 値  \\
          \hline
          train\_start  & 1 \\
          train\_end  & 400  \\
          train2\_start  & 0  \\
          train2\_end  &  0 \\
          test\_start  &  401 \\
          test\_end  &  500 \\
          k\_NN  &  38 \\
          \hline
        \end{tabular}
      \end{center}
    \end{minipage}
  \end{tabular}
\end{table}