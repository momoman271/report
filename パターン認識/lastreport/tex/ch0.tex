\section{目的}
以下の内容についてレポートにまとめる。
\subsection{課題1}
多くの従業員が勤務する工場の入退室管理をカメラで撮った画像中の顔認識で行うことになっ
た。貴方はエンジニアとして、その設計から運用開始までの責任者となった。学習データの収集、
識別器のチューニングなどを行い、十分な精度が得られると判断して運用を開始したが、正しく
従業員を認識しないというクレームが寄せられた。それに関して以下のような点について考察し
て200字以上で述べよ。
\begin{itemize}
  \item 運用前にチューニングを行って十分な精度が得られていた識別器で何が起こっていると
  言えるか
  \item それを回避するにはどのような方策が考えられるか
\end{itemize}

\subsection{課題2}
「もはやSVM(サポートベクトルマシン)を使う理由はなく、使っている人は最新の情報をキャッチ
できていない」という趣旨の若干煽り気味のツイートであるが、一理あるという反応もある。
 このツイートに関して、以下のような点について調査・考察して200字以上で述べよ。
\begin{itemize}
  \item 「もはやSVMを使用する理由はない」という理由
  \item それでもこの講義においてSVMを取り扱った理由
\end{itemize}

\subsection{課題3}
\subsubsection{課題3-1}
softmax手法で、学習率と割引率を変化させて、更新回数と方策の比較を行う。別途配布しているスクリプト(ipynb ファイル)を Google Colaboratory で実行して 20
Newsgroups dataset の識別を行うためにどの識別手法が一番適しているかを考察せよ。ここで
は、識別手法として、
\begin{itemize}
  \item k最近傍法:kNN
  \item 線形サポートベクトルマシン:linearSVM
  \item 非線形サポートベクトルマシン:nonlinearSVM
  \item ランダムフォレスト:randomForest
\end{itemize}
を候補とする。スクリプトでは他の手法も選択できるが、まずは上記のみを対象とすること。考察
の中では、
\begin{itemize}
  \item 比較対象の識別手法の評価結果(表などで分かりやすくまとめる)
  \item その結果から判断できる最良の識別手法とその理由
\end{itemize}
を述べること。ここでは第10回のスライドで説明した識別手法の選択指針に従って最良の手法を
判断すること。精度が0。05程度違うものは同じ精度であるとし、汎化性能が高くなることが見込ま
れるものを選択してその理由と共に述べること。
\subsubsection{課題3-2}
 次に、汎化性能のことは忘れて、スクリプトで得られる精度が少しでも高くなる結果を目指すこ
とにする。配布しているスクリプトでは
\begin{itemize}
  \item k最近傍法:kNN
  \item ロジスティック回帰:logistic
  \item 線形サポートベクトルマシン:linearSVM
  \item 非線形サポートベクトルマシン:nonlinearSVM
  \item 決定木:decisionTree
  \item アダブースト:adaBoost
  \item ランダムフォレスト:randomForest
\end{itemize}
が選択できる。これらすべてに対して、精度を評価して、どれが最良の精度となるか示せ。なお、
それぞれの識別手法のパラメータは標準的な設定としているが、どれか一つの手法でよいので、
標準的なパラメータから変更してみること。レポートでは以下の項目について述べること。
\begin{itemize}
  \item どの手法のどのパラメータを変更したか(可能であればその理由も)
  \item 比較対象の識別手法の評価結果(表などで分かりやすくまとめる)
  \item どれが最良の精度になったか
\end{itemize}
 さらに、Python のプログラミング知識があれば上記以外の識別手法を利用・実装しても構わな
い(ただし、深層学習を利用することは禁止する)。その場合は、どのような識別手法を利用した
かをレポートで述べること。
 
 最後に、様々な手法・パラメータを試した結果、最良と判断した識別手法とそのパラメータ設定
を提出すること。提出された識別手法・パラメータを利用してある student\_id で得られたデータで
性能評価を実施し、提出者全体の中で上位の結果となったものには若干加点する。提出方法は
以下で説明する
