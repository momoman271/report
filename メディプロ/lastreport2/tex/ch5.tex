\section{むすびと感想}
今回の実験では、胃X線像6枚を用いて、
異常胃に存在するマスクメロン模様の特徴を
同じパーツが複数存在すると捉え、どれだけ同じ特徴が存在するかで、健常胃と異常胃の分類を行った。
その結果、同じ異常胃の画像でも、写り方などの原因により、特徴の抽出が難しい画像が存在する
ことが分かった。
また、上記ことから、前処理の重要性も理解することができた。
この課題の感想:自分で1から問題を解決することの難しさに気づけて、
面白かった。