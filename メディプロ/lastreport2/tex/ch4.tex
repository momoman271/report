\section{考察}
\begin{itemize}
  \item 実験結果より、この方法においては、異常胃の画像「abnormal:3」の
  検索結果の上位3枚の画像が健常胃であることから、
  健常胃に近い特徴であることが分かった。しかし、他の異常胃の画像
  の「abnormal:1」と「abnormal:2」の距離は「14531.47」と互いに近い距離に存在していることから、
  「abnormal:3」は、他の異常胃画像に比べて、異常胃の中でも写り方が異なっていることなどが考えられる。
  \item 実験結果より、この方法においては、健常胃の画像「normal:1」
  の検索上位2枚の画像が異常胃であることから、
  異常胃に近い特徴であることが分かった。今回の方法では、異常胃に存在するマスクメロン模様の特徴を
  同じパーツが複数存在すると捉え、どれだけ同じ特徴が存在するかで識別しようとしている。
  しかし、この特徴は画素値の大小関係の比較だけであるため、例え画素値が大きく異なったとしても、
  大小関係さえ満たしていれば、同じ特徴としてみなされる。
  これにより、「normal:1」は、健常胃にも関わらず、異常胃と大小関係の構造が似ていたため
  、異常胃と識別されたと考えられる。
\end{itemize}