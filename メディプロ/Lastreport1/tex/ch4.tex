\section{考察}

\begin{itemize}
  \item 表\ref{table:2}を見ると、検索結果1位と10位の検索キーとの距離が
  「0.415」と「1.694」となっており、約4倍も変わっていることが分かる。
  これは、検索に使う特徴ベクトルが8個もあるため、距離の計算の数値の振れ幅が大きくなった
  からであると考えられる。また、画像が100枚しか存在しないため、そもそもの類似画像が少なかった
  ことも原因であると考えられる。
  \item 検索結果の第3位、第4位、第5位を見比べてみる。
    \begin{itemize}
      \item 第3位の画像は四角形があり、その下に日本語が刻まれている。
      \item 第4位の画像は円形であり、その中にアルファベットが刻まれている。
      \item 第5位の画像は下向きの矢印が3本伸びている。
    \end{itemize}
    \begin{itemize}
      \item[→] このように、文字にしてみると第3位から第5位は全く別の画像のように聞こえるものの、
      検索キーとの距離を見ると、それぞれ「0.935」、「1.088」、「1.162」と
      なっており、距離は約0.1ずつほどしか変わっていないことが分かる。
      このような結果となった原因として、今回の課題で検索に使われた特徴は類似画像の
      検索に適していない、または、計算に使う特徴が足りていないということが考えられる。
      なぜなら、人間から見ると全く違う図形であるにも関わらず、距離が近いということは、
      「人間が画像を類似しているかどうかを判断している基準」をプログラムが
      計算に含めてきれていないと考えられるからである。
    \end{itemize}
    
\end{itemize}

