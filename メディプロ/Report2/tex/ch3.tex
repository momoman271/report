\section{考察}
\subsection{課題1}
\begin{itemize}
  \item ここで計算した値は何に近づくべきなのか\\
   精度は、0から1まで存在し、1に近づけば近づくほど正しくクラスを識別できているということであるため、
  1に近づいた方が良い。
  \item 1.1.2~5分割の交差確認法で得られた精度に見られる差異とその原因\\  
   1.1.2~5分割の交差確認法で、入力したパラメータの違いは、
  学習に用いたデータとテストに用いたデータだけであり、それらの違いで見られた精度の差異については、
  モデルが学習に用いたデータの誤った挙動なども学習してしまったために、
  未知のデータであるテストデータで識別する際にも、誤った識別をしてしまっていて、
  それが学習に用いたデータとテストに用いたデータが違うだけで
  精度が変わってしまう原因になったと考えられる。
\end{itemize}

\subsection{課題2}
ハイパーパラメータを変えて、精度を出していった結果、kが1に近いときは精度が低く、そこからkを増やすほど
精度が上がっていったが、kがある値になると精度が変わらなくなり、やがて、精度が低くなることもあった。
これらのことを考察すると、kが1に近いときは、誤った挙動をしたデータなどに引っかかりやすくなったため、
精度が低くなっていると考えられ、kがある値を超えた時に精度が低くなることもあったのは、
データの偏りなどで、上手く分類できなかったと考えられる。